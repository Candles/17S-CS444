\documentclass[10pt,onecolumn,journal,draftclsnofoot,letterpaper]{IEEEtran}

\usepackage{graphicx}                                        
\usepackage{amssymb}                                         
\usepackage{amsmath}                                         
\usepackage{amsthm}                                          

\usepackage{alltt}                                           
\usepackage{float}
\usepackage{color}
\usepackage{url}

\usepackage{balance}
\usepackage{enumitem}

\usepackage[margin=0.75in]{geometry}
\geometry{textheight=8.5in, textwidth=6in}

\usepackage{longtable}

%random comment

\newcommand{\cred}[1]{{\color{red}#1}}
\newcommand{\cblue}[1]{{\color{blue}#1}}

\usepackage{hyperref}
\usepackage{geometry}

\def\name{CS444 TEAM14-03}

%pull in the necessary preamble matter for pygments output
\input{pygments.tex}

\linespread{1.0}

%% The following metadata will show up in the PDF properties
\hypersetup{
  colorlinks = true,
  urlcolor = black,
  pdfauthor = {\name},
  pdfkeywords = {CS444},
  pdftitle = {CS 444 HW1},
  pdfsubject = {CS 444 HW2},
  pdfpagemode = UseNone
  }

\begin{document}

\begin{titlepage}
  \pagenumbering{gobble}
  \title{Spring 2017 CS444\\Assignment 2 Writeup}
  \author{Chongxian Chen, Thomas Olson and Christopher Tang\\14-03}
  \date{April 21th, 2017}
  \maketitle
  \vspace{4cm}
  \begin{abstract}
  \noindent This document contains the writeup for the LOOK I/O scheduler design and implementation assignment, the Git repo history and the work log for group 14-03
 \end{abstract}
    \bigskip
    \bigskip
    \bigskip
    \bigskip
    \bigskip


%The design you plan to use to implement the SSTF algorithms.
%Your version control log.
%A work log, detailing what you did when. Ideally, this is the same as the above.
%Answer the following questions in sufficient detail:
%What do you think the main point of this assignment is?
%How did you personally approach the problem? Design decisions, algorithm, etc.
%How did you ensure your solution was correct? Testing details, for instance.
%What did you learn?

\end{titlepage}

\section{Design}

We plan on implementing the C-LOOK algorithm, as it seems to be the easier of the two options. After some research, it appears that the I/O elevator already preforms back-merging, so we do not have to implement that ourselves.

\section{Questions}

\subsection{Main point of assignment}
The idea behind creating an I/O scheduler is to help students understand that a computer is more than just an object that runs code. Writing the scheduler solidifies the fact that the processor needs to interact with other hardware in order to make it into a functional unit. Furthermore, the hardware components being interfaced with are much slower than the processor executing I/O operations and very often the hardware has mechanical limitations that need to be accounted for. These limitations require us to make optimizations on a device by device bases.

\subsection{How we approached the problem}
After reading through vast amounts of code and documentation, we decided that C-LOOK would be easier to implement than LOOK, as C-LOOK always moves in one direction while LOOK moves in both. The entire solution is implemented in about 16 lines of code, nine of which are the real important lines. The core of the algorithm is a loop that simply inserts the new request into the appropriate position in the queue, so that the queue is always properly sorted. The queue is divided into two sections: the first section contains requests with sector numbers less than or equal to the first request in the queue, and the second section contains requests with sector numbers greater than the first request in the queue. We can tell if the new request goes in the first or second half of the queue by comparing it to the value of the head, and we can tell where the insertion algorithm is currently in the queue by comparing the head to the element currently being iterated over. Once the algorithm is in the proper section of the queue, it's simply of a matter of inserting it after the first request that it is less than.

\subsection{How We Tested Our Solution}
To test our I/O solution, we created a simple program in Python to generate text files filled with random numbers. These are then pseudo-randomly accessed again to add more content. While the I/O scheduler adds these file to a queue, it will print out the sector number. If the output is a monotonically decreasing string of numbers then we know that our solution is working.

\subsection{What we learned}
We learned how scheduling algorithms work, how to compile, configure and enable Linux kernel modules and the parts of an IO scheduler.

\section{Git Log}

%% This file was generated by the script latex-git-log
\begin{tabular}{lp{12cm}}
  \label{tabular:legend:git-log}
  \textbf{acronym} & \textbf{meaning} \\
  V & \texttt{version} \\
  tag & \texttt{git tag} \\
  MF & Number of \texttt{modified files}. \\
  AL & Number of \texttt{added lines}. \\
  DL & Number of \texttt{deleted lines}. \\
\end{tabular}

\bigskip

\iflanguage{ngerman}{\shorthandoff{"}}{}
\begin{longtable}{|rlllrrr|}
\hline \multicolumn{1}{|c}{\textbf{V}} & \multicolumn{1}{c}{\textbf{tag}}
& \multicolumn{1}{c}{\textbf{date}}
& \multicolumn{1}{c}{\textbf{commit message}} & \multicolumn{1}{c}{\textbf{MF}}
& \multicolumn{1}{c}{\textbf{AL}} & \multicolumn{1}{c|}{\textbf{DL}} \\ \hline
\endhead

\endfoot

\hline% \hline
\endlastfoot

\hline 18 &  & 2017-05-06 & Simple IO generator. & 2 & 43 & 0 \\
\hline 19 &  & 2017-05-07 & Added responce to writeup + second pass to iogen & 2 & 14 & 4 \\
\hline 20 &  & 2017-05-07 & Finished sstf-iosched.c; updated configand makefile; added python iogen & 4 & 263 & 41 \\
\hline 21 &  & 2017-05-07 & Finished concurrency assignment & 1 & 69 & 10 \\
\hline 22 &  & 2017-05-08 & Added scheduler patch & 1 & 235 & 0 \\
\end{longtable}


\section{Work Log}

\subsection{May 3rd}
Started concurrency assignment

\subsection{May 5th}
Finished concurrency assignment

\subsection{May 6th}
Started kernel assignment

\subsection{May 7th}
Finished sstf-iosched.c, modified config files and started initial testing of C-LOOK scheduler

\subsection{May 8th}
Finished testing kernel assignment and documentation

\end{document}











































