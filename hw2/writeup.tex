\documentclass[10pt,onecolumn,journal,draftclsnofoot,letterpaper]{IEEEtran}

\usepackage{graphicx}                                        
\usepackage{amssymb}                                         
\usepackage{amsmath}                                         
\usepackage{amsthm}                                          

\usepackage{alltt}                                           
\usepackage{float}
\usepackage{color}
\usepackage{url}

\usepackage{balance}
\usepackage{enumitem}

\usepackage[margin=0.75in]{geometry}
\geometry{textheight=8.5in, textwidth=6in}

\usepackage{longtable}

%random comment

\newcommand{\cred}[1]{{\color{red}#1}}
\newcommand{\cblue}[1]{{\color{blue}#1}}

\usepackage{hyperref}
\usepackage{geometry}

\def\name{CS444 TEAM14-03}

%pull in the necessary preamble matter for pygments output
\input{pygments.tex}

\linespread{1.0}

%% The following metadata will show up in the PDF properties
\hypersetup{
  colorlinks = true,
  urlcolor = black,
  pdfauthor = {\name},
  pdfkeywords = {CS444},
  pdftitle = {CS 444 HW1},
  pdfsubject = {CS 444 HW2},
  pdfpagemode = UseNone
  }

\begin{document}

\begin{titlepage}
  \pagenumbering{gobble}
  \title{Spring 2017 CS444\\Assignment 2 Writeup}
  \author{Chongxian Chen, Thomas Olson and Christopher Tang\\14-03}
  \date{April 21th, 2017}
  \maketitle
  \vspace{4cm}
  \begin{abstract}
  \noindent This document contains the writeup for the LOOK I/O scheduler design and implementation assignment, the Git repo history and the work log for group 14-03
 \end{abstract}
    \bigskip
    \bigskip
    \bigskip
    \bigskip
    \bigskip


%The design you plan to use to implement the SSTF algorithms.
%Your version control log.
%A work log, detailing what you did when. Ideally, this is the same as the above.
%Answer the following questions in sufficient detail:
%What do you think the main point of this assignment is?
%How did you personally approach the problem? Design decisions, algorithm, etc.
%How did you ensure your solution was correct? Testing details, for instance.
%What did you learn?

\end{titlepage}

\section{Design}

\section{Questions}

\subsection{Main Point Of Implementing the LOOK I/O Scheduler}
The idea behind creating an I/O scheduler is to help students understand that a computer is more than just an object that runs code. Writing the scheduler solidifies the fact that the processor needs to interact with other hardware in order to make it into a functional unit. Furthermore, the hardware components being interfaced with are much slower than the processor executing I/O operations and very often the hardware has mechanical limitations that need to be accounted for. These limitations require us to make optimizations on a device by device bases.

\subsection{How We Approached the Problem}

\subsection{How We Tested Our Solution}

\section{Git Log}

\section{Work Log}

\end{document}











































