\documentclass[10pt,onecolumn,journal,draftclsnofoot,letterpaper]{IEEEtran}

\usepackage{graphicx}                                        
\usepackage{amssymb}                                         
\usepackage{amsmath}                                         
\usepackage{amsthm}                                          

\usepackage{alltt}                                           
\usepackage{float}
\usepackage{color}
\usepackage{url}

\usepackage{balance}
\usepackage{enumitem}

\usepackage[margin=0.75in]{geometry}
\geometry{textheight=8.5in, textwidth=6in}

\usepackage{longtable}

%random comment

\newcommand{\cred}[1]{{\color{red}#1}}
\newcommand{\cblue}[1]{{\color{blue}#1}}

\usepackage{hyperref}
\usepackage{geometry}

\def\name{CS444 TEAM14-03}

%pull in the necessary preamble matter for pygments output
\input{pygments.tex}

\linespread{1.0}

%% The following metadata will show up in the PDF properties
\hypersetup{
  colorlinks = true,
  urlcolor = black,
  pdfauthor = {\name},
  pdfkeywords = {CS444},
  pdftitle = {CS 444 HW3},
  pdfsubject = {CS 444 HW3},
  pdfpagemode = UseNone
  }

\begin{document}

\begin{titlepage}
  \pagenumbering{gobble}
  \title{Spring 2017 CS444\\Assignment 3 Writeup}
  \author{Chongxian Chen, Thomas Olson and Christopher Tang\\14-03}
  \date{April 21th, 2017}
  \maketitle
  \vspace{4cm}
  \begin{abstract}
  \noindent This document contains the writeup for the Encrypted Block Device implementation assignment, the Git repo history and the work log for group 14-03
 \end{abstract}
    \bigskip
    \bigskip
    \bigskip
    \bigskip
    \bigskip


%The design you plan to use to implement the SSTF algorithms.
%Your version control log.
%A work log, detailing what you did when. Ideally, this is the same as the above.
%Answer the following questions in sufficient detail:
%What do you think the main point of this assignment is?
%How did you personally approach the problem? Design decisions, algorithm, etc.
%How did you ensure your solution was correct? Testing details, for instance.
%What did you learn?

\end{titlepage}

\section{Design}

We plan on implementing the C-LOOK algorithm, as it seems to be the easier of the two options. After some research, it appears that the I/O elevator already preforms back-merging, so we do not have to implement that ourselves.

\section{Questions}

\subsection{Main point of assignment}


\subsection{How we approached the problem}


\subsection{How We Tested Our Solution}


\subsection{What we learned}



\section{Git Log}



\section{Work Log}


\end{document}











































