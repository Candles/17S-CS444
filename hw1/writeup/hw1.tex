\documentclass[letterpaper,10pt,titlepage]{article}

\usepackage{graphicx}                                        
\usepackage{amssymb}                                         
\usepackage{amsmath}                                         
\usepackage{amsthm}                                          

\usepackage{alltt}                                           
\usepackage{float}
\usepackage{color}
\usepackage{url}

\usepackage{balance}
\usepackage[TABBOTCAP, tight]{subfigure}
\usepackage{enumitem}
\usepackage{pstricks, pst-node}

\usepackage{geometry}
\geometry{textheight=8.5in, textwidth=6in}

%random comment

\newcommand{\cred}[1]{{\color{red}#1}}
\newcommand{\cblue}[1]{{\color{blue}#1}}

\usepackage{hyperref}
\usepackage{geometry}

\def\name{CS444 TEAM14-03}

%pull in the necessary preamble matter for pygments output
\input{pygments.tex}

%% The following metadata will show up in the PDF properties
\hypersetup{
  colorlinks = true,
  urlcolor = black,
  pdfauthor = {\name},
  pdfkeywords = {CS444},
  pdftitle = {CS 444 HW1},
  pdfsubject = {CS 444 HW1},
  pdfpagemode = UseNone
  }

\begin{document}

\begin{center}

CS444
\bigbreak
HW1
\bigbreak
By Thomas Olson, Christopher Tang, Chongxian Chen
\end{center}

\section{Command logs}
\
\par First we login to our os-server by logging into our filp server first then "ssh os-class"
\par mkdir /scratch/spring2017/14-03
\par cd /scratch/spring2017/14-03
\par git clone�git://git.yoctoproject.org/linux-yocto-3.14
\par cd linux-yocto-3.14
\par git checkout v3.14.26
\par We are using Tcsh, so we ran "source /scratch/opt/environment-setup-i586-poky-linux.csh"
\par cp /scratch/spring2017/files/config-3.14.26-yocto-qemu .config
\par make menuconfig
\par press / and type in LOCALVERSION, press enter.�
\par Hit 1, press enter and then edit the value to be -14-03-hw1
\par make -j4 all
\par cd ..
\par gdb
\par in a new terminal, cd /scratch/spring2017/14-03
\par cp /scratch/spring2017/files/bzImage-qemux86.bin .
\par cp /scratch/spring2017/files/core-image-lsb-sdk-qemux86.ext3 . 
\par qemu-system-i386 -gdb tcp::5643 -S -nographic -kernel bzImage-qemux86.bin -drive file=core-image-lsb-sdk-qemux86.ext3,if=virtio -enable-kvm -net none -usb -localtime --no-reboot --append "root=/dev/vda rw console=ttyS0 debug"
\par In gdb, target remote :5643
\par continue
\par login as root
\par uname -a
\par reboot
\par qemu-system-i386 -gdb tcp::5643 -S -nographic -kernel linux-yocto-3.14/arch/x86/boot/bzImage� -drive file=core-image-lsb-sdk-qemux86.ext3,if=virtio -enable-kvm -net none -usb -localtime --no-reboot --append "root=/dev/vda rw console=ttyS0 debug"
\par iin gdb, target remote :5643
\par continue
\par login as root
\par uname -a
\par reboot
\par q to quit gdb


\section{Qemu Flags}

\end{document}